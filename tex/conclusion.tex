\chapter{Conclusions and Recommendations}
% -------------------------
%% QUOTE
\vspace*{\fill}
\epigraph{There must be a beginning of any great matter, but the continuing unto the end until it be thoroughly finished yields the true glory.}%
{\textit{Letter to Sir Francis Walsingham, from off Cape Sagres, Portugal (17 May 1587)} \textsc{Sir Francis Drake}}
\clearpage{\thispagestyle{empty}\cleardoublepage}
%%
%% Body of the chapter
%%%%%%%%%%%%%%%%%%%%%%
Deep placement of gel in water flooded oil reservoirs may block channels with high water flow and divert the water into other parts of the reservoir. In order to get the gel constituents to the correct positions a delay in the gelling time up to weeks or months at elevated temperatures will be necessary. Further, the gel constituents must be environmentally acceptable. The objective of the HyGreGel project was to develop gel systems based on innovative chemistry taking nanoparticles into use and where the gelling time can be increased. Further, mechanisms for reaction and transport of the gel constituents were described enabling modelling of deep gel placement in oil reservoirs.

Transport of deactivated FN particles and polymer in porous media was studied and described through several core flooding experiments. Key properties like adsorption, total retention, inaccessible pore volume and permeability reduction were determined for two Bentheimer and Berea sandstones. The results indicated that parameters like polymer adsorption, polymer retention and IPV were not significantly affected by the presence of nanoparticles. Polymer retained during injection appeared not to be released during following water floods.

Use of polyelectrolyte complexes as a method for delayed gelation based on published
technology developed at Texas A\&M University was tested. Delayed gelation was obtained at 50~\celsius~ but not at 80 ℃. When the original polycation was replaced  
active FN particles (which are also polycations), and dextranesulphate was replaced with polyvinylsulphonate, a reduced rate of gelation at 80~\celsius~ was also obtained.  

A methodology for controlled gelation of partially hydrolyzed polyacrylamide (HPAM) using hybrid nanoparticles with functional groups (FN particles) as crosslinkers was developed. Several systems with different types of FN particles were synthesized, characterized and tested. Both FN particles with fully blocked reactive sites that were slowly activated by hydrolysis and partially deactivated FN particles could significantly delay the gelation rate, giving gelling times ranging from several days to several months in SSW at 80~\celsius.

A gel system based on partially modified FN particles was tested in core flooding experiments. The resistance to water flow increased with aging time, ultimately giving a very strong gel. The chemical system used has a severe injectivity problem. This problem was solved by an improved treatment consisting of better filtering and reduced pH. Reducing pH appeared to decrease the amount of precipitates in POSS and gel systems (with POSS and HPAM).    

A mathematical model of 1D two-phase flow with polymers and nanoparticles was developed, implemented and tested. The formulation includes tracking the age and concentration distribution of the nanoparticles. This means that at each position and time, the composition of the nanoparticles with respect to age and concentration are given. The age composition of the particles in solution in turn affects the nanoparticles ability to crosslink and form gel. Results from the core flooding experiments with deactivated FN particles and polymer were used in the testing the model. The model generated a good fit of the experimental results. A good match was also achieved by comparing the results from the developed model with those of a commercial simulator, ECLIPSE. This model was then used to simulate gelation in field scale case.

The use of a selected type of deactivated FN particles for improving oil recovery during water flooding was tested in both water wet and neutral wet Berea cores. No improvements in the oil recovery were obtained, however.

A significant step towards enabling hybrid material technology as a next generation of chemical EOR has been taken through the HyGreGel project. However, within the frame of the present project it has not been possible to follow all the topics involved in the technology to the end. Further research and development will be needed to optimize and improve the technology to the level needed for field testing.  
