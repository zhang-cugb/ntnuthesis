% DEDICATION
\begin{dedication}
To my parents, who inspired me to commence the PhD;\\
To my sister, Hanieh, who propelled me to  carry on with it;\\
To my love, Helia, who motivated me to bring it to a conclusion.
\end{dedication}

% PREFACE
\clearpage{\thispagestyle{empty}\cleardoublepage}
\setcounter{page}{1}
\chapter*{Preface}

This thesis is submitted to the Norwegian University of Science and Technology (NTNU) for partial fulfillment of the requirements for the degree of Philosophiae Doctor.

The work presented in this thesis was conducted at the Department of Geoscience and Petroleum (IGP), NTNU, Trondheim. Professor Ole Torsæter was the main supervisor. Dr. Torleif Holt and Dr. Jan Åge Stensen from SINTEF Industry were the co-supervisors. Testing infrastructures at SINTEF Industry, NTNU IGP and NTNU NanoLab were used in the experimental work.  

The research was funded by The Research Council of Norway, SINTEF, NTNU, Aker BP ASA, Engie E\&P Norge AS, Eni Norge AS and Lundin Norway AS.


% ABSTRACT
\clearpage{\thispagestyle{empty}\cleardoublepage}
\chapter*{Abstract}

High water production is typically a major problem late in the life cycle of a water flooded hydrocarbon reservoir. Reservoir heterogeneity plays a significant role in creating this problem. An example is highly permeable zones and streaks in the reservoir, the so-called ``thief zones". These zones attract the injected water and result in early water breakthrough, hence high water cuts in production wells. 

One solution to this problem is blocking the thief zones. As a result, the injected water will be diverted from channels with high water flow into other (potentially oil-bearing) areas of the reservoir. The prospective end result is reduced water production and increased oil production.  This may be achieved by deep placement of polymer gels in the reservoir. 

Deep placement of gels is challenging. This is owing to the fact that fully formed gels cannot be transported through porous media deep into the formation. A workaround is to transport gel constituents, \textit{i.e.}, polymer and a cross-linking agent, to the proper area in the reservoir before gelling begins. This requires a delay in gelling time at elevated temperatures of the reservoir. Moreover, the gel constituents must be environmentally sound.

This PhD work was part of the HyGreGel (Hybrid Green Gels) project. As a result of this project gel systems with delayed gelation times were developed and tested. The objective was to enable the chemicals to reach the desired position in the reservoir before gelation would prohibit further transport. Furthermore, mechanisms for transport and reaction of the gel constituents were described and modelled. 


% ACKNOWLEDGEMENT
\clearpage{\thispagestyle{empty}\cleardoublepage}
\chapter*{Acknowledgement}
{\centering {\Huge \colorbox{Red}{\color{white}!}}\par}


% NOMENCLATURE
\clearpage{\thispagestyle{empty}\cleardoublepage}
\nomenclature[A]{EOR}{Enhanced Oil Recovery}
\nomenclature[A]{NTNU}{Norwegian University of Science and Technology}
\nomenclature[A]{SSW}{Synthetic sea water}

\printnomenclature

