\chapter{Introduction}
% -------------------------
%% QUOTE
\vspace*{\fill}
\epigraph{In all matters, before beginning,\\ a diligent preparation should be made.}%
{\textit{De Officiis (44 B.C.), I. 21.}\\ \textsc{Cisero}}
\clearpage{\thispagestyle{empty}\cleardoublepage}
%%
%% Body of the chapter
%%%%%%%%%%%%%%%%%%%%%%
\section{Motivation}

Challenges and technology gaps within the area of Enhanced Oil Recovery (EOR), \index{EOR!challenges} which need to be addressed in the coming research programs, have been identified in the OG21 strategy document on ``Exploration and Increased Recovery" \citep{OG21}: In particular, there is a need for more cost-efficient EOR chemicals, and to assure environmentally acceptable methods to avoid unwanted discharge to sea. Improved petroleum resource exploitation by EOR has also been given special attention in a report on increased recovery in the Norwegian Continental Shelf by the Norwegian Department of Oil and Energy \citep{Am2010}. There is therefore a clear need for increased competence in new technologies in the field of EOR. 

There has recently been an increasing interest in applying nanotechnology to EOR but still many topics are uncovered. Nanotechnology, \index{Nanotechnology} which has mainly been developed in mechanical engineering, medicine and biological sciences, is expected to have a large potential for EOR applications. This technology has a wide range of applications relevant to EOR, from employing general concepts and principles of nanotechnology, to advanced reservoir monitoring using nano-sensors and nano-analysis, to more specific technologies which make up for the shortcomings in traditional EOR methods \citep{Fletcher2010, Ayatollahi2012, Cocuzza2011}. 

\colorbox{red}{The latter includes tailoring chemical molecules for more efficient EOR, and smart and more efficient delivery of the EOR agent. Development of efficient drug delivery in the human body is an example where such technology may be applied to improve efficiency in chemical flooding.}


