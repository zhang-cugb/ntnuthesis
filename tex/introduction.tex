\chapter{Introduction}
% -------------------------
%% QUOTE
\vspace*{\fill}
\epigraph{In all matters, before beginning,\\ a diligent preparation should be made.}%
{\textit{De Officiis (44 B.C.), I. 21.}\\ \textsc{Cisero}}
\clearpage{\thispagestyle{empty}\cleardoublepage}
%%
%% Body of the chapter
%%%%%%%%%%%%%%%%%%%%%%
\section{Motivation}
The surge in energy consumption levels is inevitable for modern societies. Their sustainability depends on it. The \usebibentry{EIA2017}{title} report \citeyearpar{EIA2017} published by the \usebibentry{EIA2017}{institution} projects a 28\% increase in world energy consumption between 2015 and 2040, assuming continuous improvement in known technologies based on current trends. As shown in Figure \ref{cht:energySources}, the consumption of ``petroleum and other liquid fuels" alone will grow by 18\%, which will account for 31\% of total world energy consumption in 2040.

\begin{figure}[b!]
    \centering
    \includegraphics[width=\textwidth]{img/chart/chtEiaEnergy}
    \caption{World energy consumption by energy source \citep{EIA2017}}
    \label{cht:energySources}
\end{figure}

It is becoming more and more challenging to meet the ever-increasing demand for petroleum. Most of the existing major oilfields are already at a mature stage and the number of new significant discoveries per year is decreasing. Therefore, at this point in time, it is crucial to focus on methods of improving petroleum production from existing reservoirs. A big subset of such methods fall under the category of Enhanced Oil Recovery (EOR).

Challenges and technology gaps within EOR, \index{EOR!challenges} which need to be addressed in the coming research programs, have been identified in the OG21 strategy document on ``Exploration and Increased Recovery" \citep{OG21}: In particular, there is a need for more cost-efficient EOR chemicals, and to assure environmentally acceptable methods to avoid unwanted discharge to sea. Improved petroleum resource exploitation by EOR has also been given special attention in a report on increased recovery in the Norwegian Continental Shelf by the Norwegian Department of Oil and Energy \citep{Am2010}. There is therefore a clear need for increased competence in new technologies in the field of EOR. 

There has recently been an increasing interest in applying nanotechnology to EOR but still many topics are uncovered. Nanotechnology, \index{Nanotechnology} which has mainly been developed in mechanical engineering, medicine and biological sciences, is expected to have a large potential for EOR applications. This technology has a wide range of applications relevant to EOR, from employing general concepts and principles of nanotechnology, to advanced reservoir monitoring using nano-sensors and nano-analysis, to more specific technologies which make up for the shortcomings in traditional EOR methods \citep{Fletcher2010, Ayatollahi2012, Cocuzza2011}. The latter includes tailoring chemical molecules for more efficient EOR, as well as smart and more effective delivery of EOR agents. Such technologies as efficient drug delivery in the human body may be applied to improve efficiency in chemical flooding.

This PhD work is part of the HyGreGel\index{HyGreGel|(} (Hybrid Green nano-Gels) project. The HyGreGel project addresses the need for more efficient water diversion techniques by improved in-depth placement of gelling chemicals to increase waterflood recovery and reduce unwanted water circulation in heterogeneous reservoirs. The project recognizes the environmental challenges using chemicals and emphasizes the development of green chemical systems for such applications.
\section{Problem formulation}

Incremental recovery from water diversion is generally expected to be merely 2\% above standard water flooding \citep{OG21}. The main objective of HyGreGel is thus 
\begin{tcolorbox}
to improve incremental recovery from water diversion by developing and testing innovative hybrid (polymer + nanoparticle) gels.
\end{tcolorbox}
A thorough study of the effect of surface functionalities of the nanoparticles and their reactivity with the polymers will enable great improvements in controlling gel formation. This can be achieved by learning the scientific reasons behind the experimental results and understanding the kinetics of transport of nano-gels through porous media. Hence,  HyGreGel will contribute to making nanotechnology the next-generation EOR method.

The sub-goals of HyGreGel are:
\begin{enumerate}
    \item To study the effect of functional nanosized particles on the gel formation with polymers by cross-linking, and their transport through the oil reservoir.
    \item To synthetize and further develop  hybrid materials, both in terms of functionality and size.
    \item To investigate the reactivity of the nanosized particles with polymers, ensuring controlled gelling for several weeks and possibly months to assure in-depth placement of gels.
    \item To create basic knowledge on controlled release of encapsulated active components.
    \item To study the effect of real reservoir parameters such as temperature, pressure, brine salinity, and presence of crude oil on gel strength and gel stability and optimize the operational parameters.
    \item To study the integration of hybrid materials in industrial EOR.
\end{enumerate}

The results from HyGreGel will provide an understanding of how the nanoparticles affect the gel properties and their transport mechanism through the oil reservoir. They will also take major steps toward a robust design and manufacturing of hybrid gels. HyGreGel will keep the advantages of polymer materials (flexibility, processability and low cost) and make some improvement in EOR by integrating hybrid nanoparticles in the polymer. \index{HyGreGel|)}

\section{Sub-projects}
The scientific work in the project was done through four sub-projects (SP). The research tasks and scientific methods for each of the SPs are described below.

\subsection*{SP1 --- Hybrid materials as sweeping efficiency modifier}
\addcontentsline{toc}{section}{\protect\numberline{}SP1 \quad Hybrid materials as sweeping efficiency modifier}%

SP1 was performed at the Department of Materials and Nanotechnology at SINTEF Industry (DMN). The approach is to use FunzioNano\texttrademark (FN)\index{FunzioNano} particles with functional groups having the ability to react with polymers causing cross-binding and gelling. Figure \ref{fig:sp1sp2} shows the gelation mechanism for SP1.

Hybrid materials based on FunzioNano are multifunctional nanoparticles which can provide significant benefit in EOR. FunzioNano with blocked functional groups can be used as a latent cross-linker for polymers which are made from renewable resources, e.g. cellulose based polymers. 

FunzioNano cross-linker and polymer are injected as a formulation at the same time. Hydrolysis within the requested time window leads to de-blocking of the functional groups and thereafter fast and efficient gelation by cross-linking of FunzioNano and polymer, e.g. through ionic bond formation between amine functionalities on FunzioNano and carboxylic functionalities on cellulose.

The time window for de-blocking FunzioNano can be adjusted by the type and the amount of hydrolysis catalyst. As an additional benefit cross-linked FunzioNano and polymer can provide a more hydrophobic gel than the components themselves. Motion of water could therefore be more efficiently hindered than motion of produced oil which would be suitable for partially blocking gels that can transport the remaining oil. FunzioNano can to a significant extent be manufactured from renewable resources. In case of un-desired spill, FunzioNano is degradable after being highly diluted with water. Computer modelling of the degradability of highly diluted FunzioNano was previously performed in collaboration with Université de Savoie, France \citep{Neyertz2012,Neyertz2013}.

\subsection*{SP2 --- Nanogels from polyelectrolyte complexes}
\addcontentsline{toc}{section}{\protect\numberline{}SP2 \quad Nanogels from polyelectrolyte complexes}

SP2 was performed at the Department of Exploration and Reservoir Technology (DER) and DMN at SINTEF Industry in co-operation with researchers at Texas A\&M University.

The task is to encapsulate active chemicals like cross-linkers by using polyelectrolyte complex \index{Polyelectrolyte Complexes} (PEC) nanoparticles to form a carrier which can be transported into a porous medium at relevant reservoir conditions. The hybrid materials developed in SP1 can be used as active chemicals to be encapsulated in PEC nanoparticles. 

Figure \ref{fig:sp1sp2} illustrates how hybrid systems contribute to form gel deep in the reservoir. The main challenge is to be able to control the release of the active components for correct location of the gel in the formation at conditions typical for North Sea reservoirs. The rheological properties (gelation rate, gel strength, stability) of the chemical gel system is studied in fluid bulk studies. The release of chemicals is analyzed, and factors which may influence the gelling rate is evaluated. Relevant pore scale mechanisms for the actual chemical system will be studied and analyzed.

\begin{figure}
    \centering
    \includegraphics[width=\textwidth]{img/fig/sp1sp2.png}
    \caption{Gelation mechanisms in SP1 and SP2}
    \label{fig:sp1sp2}
\end{figure}

\subsection*{SP3 --- Transport of nanoparticles}
\addcontentsline{toc}{section}{\protect\numberline{}SP3 \quad Transport of nanoparticles}%
SP3 was performed at DER SINTEF Industry and the Department of Geoscience and Petroleum (IGP) NTNU.  
The constituents of the gelling chemicals should be easily transported together with the injected water without significant loss through the porous medium to the predetermined location where the gels are to be formed. The chemicals should withstand the actual reservoir conditions in the formation (temperature, pressure, salinity, presence of reservoir fluids, etc.), and the release of active chemicals should be controlled for correct in-depth placement of the gel structure. 

The placement of water diverting gels will be field specific depending on the reservoir geology and the actual performance of the water flood process. An in-depth gel treatment will require a thoroughly investigation and analysis of the reservoir water flood to define an optimal location of the gel. 

Flooding studies with injection of chemicals is performed in order to investigate transport properties of the chemical systems involved. The porous media selected for flooding studies need to be well characterized. The effect of water salinity on transport properties is studied by a set of core flooding studies (also in the presence of crude oil). In a field case the injected fluids are exposed to reservoir temperatures and pressures, and dedicated core flood studies will be performed to evaluate how temperature and pressure will affect the transport properties and gelling of the chemical system. Chemical analysis of the produced fluids is performed in order to characterize transport properties of the various components.

\subsection*{SP4 --- Numerical simulation of the hybrid gel/polymer interaction with water/oil/soil}
\addcontentsline{toc}{section}{\protect\numberline{}SP4 \quad Numerical simulation of the hybrid gel/polymer interaction with water/oil/soil}%

SP4 was performed as a co-operation between the Department of Exploration and Reservoir technology at SINTEF and the Department of Industrial Process Technology at SINTEF Industry. Two approaches were followed for modeling the gelation phenomenon. 

\textbf{Approach 1} --- A mathematical model of 1D two-phase flow with polymers and nanoparticles is developed, implemented and tested. Results from the core flooding experiments with deactivated FN particles and polymer (from SP3) is used in the testing.

Simulations are run using a synthetic data set where polymer viscosity depends on polymer concentration and the concentration and age of the injected nanoparticles (the time passed since injection).

The formulation includes tracking the age distribution of the nanoparticles in solution as well as the age distribution of absorbed nanoparticles. This means that at each position and time, the composition of the nanoparticles with respect to age and concentration are given. The age composition of the particles in solution in turn affects the nanoparticles ability to cross-link and form gel. Consequently, this formulation facilitates simulation of gel forming far away from the injection point. The simulator also includes the standard effects of polymer on water mobility, such as shear thinning, permeability reduction due to adsorbed polymer, and inaccessible pore space for polymers and nanoparticles.

The numerical formulation is first order upwind implicit for saturation, polymer, and nanoparticles. The tracking of the age distributions uses a novel explicit upstream, implicit downstream formulation giving accuracy (limited dispersion) and stability. The linear solver is also efficient and explicit (non-iterative), and the simulator uses an automatic time step control. 

\textbf{Approach 2} --- In a second approach, an alternative 1D model is established for the transport of the polymer and nanoparticles during flooding situations. Also in this model, the nanoparticles serve as cross-linking agent. A constitutive equation is formulated based on the gelation data for a given set of polymer concentrations, a given type of nanoparticles, and measured gelation times. In the model, a residence time for the nanoparticles is used to link the viscosity of the cross-linked polymer as the mixture is transported. The model has been used for simplified simulations of injection of time delayed gelling polymers to illustrate the functionality of the model and to carry out a sensitivity analysis.

\section{Contributions and scope of the PhD work}

As mentioned earlier, this PhD work is a subset of the greater HyGreGel project. Therefore, it is important to clearly identify the scope of this work. The primary contributions are as follows:

\begin{enumerate}
    \item (SP1) The first gelation tests at DER with active FN particles and polymer revealed that it was difficult to form gels with cellulose based polymer but strong gels were formed with partially hydrolysed polyacrylamide, HPAM. A low molecular weight HPAM was chosen as the base polymer for the following studies. DER followed two approaches in synthesizing FN-based particles. 
    
    Approach 1 --- The reactive groups of the FN particles were blocked to prevent instant cross binding with the polymer. Then, as a time dependent event, the blocked sites were activated by slow hydrolysis enabling cross-linking, viscosity increase and eventually gel formation.
    
    Approach 2 --- Only parts of the active sites were deactivated. DER found that by partially deactivation of the active sites the formation of gel was controlled to vary from several days to several months.
    
    \textbf{Contribution:} Both types of FN particles synthesized by DER were tested in bulk conditions. Their reactivity was tested by measuring viscosity and gel formation as a function of time at 80~\celsius~ in synthetic sea water at anaerobic conditions. Tested systems were kept in reaction vials and eventually the most promising system  was selected for \emph{in situ} gelling experiments.
    
    \item Polymer and nanoparticle transport was investigated through core flooding tests with several configurations.
    \item A numerical model developed at SINTEF industry was first tuned to match the results from experiments and thereafter used to simulate transport and gelation of particles in field scale
\end{enumerate}
