\chapter{Introduction}
% -------------------------
%% QUOTE
\vspace*{\fill}
\epigraph{In all matters, before beginning,\\ a diligent preparation should be made.}%
{\textit{De Officiis (44 B.C.), I. 21.}\\ \textsc{Cisero}}
\clearpage{\thispagestyle{empty}\cleardoublepage}
%%
%% Body of the chapter
%%%%%%%%%%%%%%%%%%%%%%
\section{Motivation}

Challenges and technology gaps within the area of enhanced oil recovery (EOR) which need to be addressed in the coming research programs have been identified in the OG21 strategy document on "Exploration and Increased Recovery" (OG 2011). Especially, there is a need for more cost-efficient EOR chemicals, and to assure environmental acceptable methods to avoid unwanted discharge to sea. There is therefore a clear need for increased competence on new technologies in the field of EOR. There has recently been an increasing interest in applying nanotechnology to EOR but still many topics are uncovered. Nanotechnology, which has mainly been developed in mechanical engineering, medicine and biological sciences, is expected to have a large potential for EOR applications. There is a wide range of applications of this technology, which is relevant for EOR, from employing general concepts and principles from nanotechnology, to advanced reservoir monitoring using nano-sensors and nano-analysis and to more specific technologies to overcome shortcomings in traditional EOR methods. The latter includes tailoring chemical molecules for more efficient EOR, and smart and more efficient delivery of the EOR agent. Development of efficient drug delivery in the human body is an example where such technology may be applied to improve efficiency in chemical flooding.


