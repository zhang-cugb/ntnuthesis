% % DEDICATION
\begin{dedication}
To my parents, who inspired me to set out on a PhD journey,\\
to my sister, Hanieh, who propelled me to carry on with it,\\
and to my love, Helia, who encouraged me to bring it to a conclusion.
\end{dedication}

% % PREFACE
\clearpage{\thispagestyle{empty}\cleardoublepage}
\setcounter{page}{1}
\chapter*{Preface}

This thesis is submitted to the Norwegian University of Science and Technology (NTNU) for partial fulfillment of the requirements for the degree of Philosophiae Doctor.

The work presented in this thesis was conducted at the Department of Geoscience and Petroleum (IGP), NTNU, Trondheim. Professor Ole Torsæter was the main supervisor. Dr. Torleif Holt and Dr. Jan Åge Stensen from SINTEF Industry were the co-supervisors. Testing infrastructures at SINTEF Industry, NTNU IGP and NTNU NanoLab were used in the experimental work.  

The research was funded by The Research Council of Norway, SINTEF, NTNU, Aker BP ASA, Engie E\&P Norge AS, Vår Energi and Lundin Norway AS.


% ABSTRACT
\clearpage{\thispagestyle{empty}\cleardoublepage}
\chapter*{Abstract}

High water production is typically a major problem late in the life cycle of a water flooded hydrocarbon reservoir. Reservoir heterogeneity plays a significant role in creating this problem. An example is highly permeable zones and streaks in the reservoir, the so-called ``thief zones". These zones attract the injected water and result in early water breakthrough, hence high water cuts in production wells. 

One solution to this problem is blocking the thief zones. As a result, the injected water will be diverted from channels with high water flow into other (potentially oil-bearing) areas of the reservoir. The prospective end result is reduced water production and increased oil production.  This may be achieved by deep placement of polymer gels in the reservoir. 

Deep placement of gels is challenging. This is owing to the fact that fully formed gels cannot be transported through porous media deep into the formation. A workaround is to transport gel constituents, \textit{i.e.}, polymer and a cross-linking agent, to the proper area in the reservoir before gelling begins. This requires a delay in gelling time at elevated temperatures of the reservoir. Moreover, the gel constituents must be environmentally sound.

This PhD work was part of the HyGreGel (Hybrid Green nano-Gels) project. As a result of this project gel systems with delayed gelation times were developed and tested. The objective was to enable the chemicals to reach the desired position in the reservoir before gelation would prohibit further transport. Furthermore, mechanisms for transport and reaction of the gel constituents were described and modelled. 

 
% ACKNOWLEDGEMENT
\clearpage{\thispagestyle{empty}\cleardoublepage}
\chapter*{Acknowledgement}

This work would not be accomplished without the help and support of so many people.

Thank you to my advisor, Professor Ole Trosæter, NTNU, for trusting me with this topic. I am thankful for all the help and support you provided me during the past few years for my PhD study. Carrying out this project has been such a unique experience, and if not for your continuous motivation and encouragement, especially at the harder times, I would still be daydreaming about when the thesis would be completed. I am grateful for having you as my advisor.   

A special thank you to my co-advisor Dr.\ Torleif Holt, SINTEF Industry, for the continuous support of my PhD research, for your patience, motivation, and immense knowledge. Your meticulous guidance helped me in all the time of research and writing of the papers. I could not have imagined having a better advisor and mentor for my PhD study. Working as your protégé in the lab has been one of the most exceptional episodes of my life. I not only got to learn so much from your laboratory expertise, but also had to level up my broken Norwegian to appreciate your wit. I am simply thankful for having been involved in a project with you. 

Thank you to Dr.\ Jan Åge Stensen, my co-advisor at SINTEF Industry, for your continuous support with resources and arranging and facilitating productive meetings with the project's steering committee. Thank you to Dr.\ Christian Simon and Dr.\ Juan Yang, SINTEF Industry, for supporting the project by synthesizing the nanomaterials and providing content and detailed feedback for the publications.

A special thank you to Professor Dag Wessel-Berg, NTNU, for your continuous support of the project by developing a novel simulator capturing the nuances of our research. I appreciate your patience, thoroughness and fine sense of humor in explaining the finer details of developing such a simulator from scratch. I am also grateful for your detailed feedback throughout the different phases of the project. Also thanks for covering for me at the conference in Cancun while I was busy getting deported.

Thank you to Dr.\ Per Bergmo, SINTEF Industry, for your ideas and advice which greatly enriched this research project. I appreciate your support with testing the in-house and commercial simulators. Thank you to the other members of the steering committee Drs.\ Mohammed Mhamdi of SINTEF Industry, Bjørn Gulbrandsen of Lundin, Trygve Nilsson of Aker BP, Siroos Salimi of Vår Energi and Mailin Seldal of ENGIE. A continuous dialog with you through the steering committee meetings and receiving feedback and ideas from you definitely enhanced the quality of the project. I appreciate the financial support of the Research Council of Norway and the oil companies Vår Energi, ENGIE, Det Norske Oljeselskap ASA and Lundin Norway AS.

During my PhD study, I had great pleasure supervising the course Experts in Teamwork (Norne Village) at the Department of Geoscience and Petroleum at NTNU. A big thank you to Professors Egil Tjåland and Jon Kleppe for supporting me with this task for four years. It was an honor for me to rub shoulders with you in this position. Also thank you for choosing me as part of the Norwegian delegacy at the Islamabad Conference in 2016. Traveling and exploring together with you was such a delightful experience, except maybe the part were we all had to deal with the food poisoning.    

Throughout the PhD, I got to make many friends and mingle with great colleagues. Thank you to Hamid, Marco, Cleide, Nan, Vegard, Filipe, Ashkan, Tore, Rasoul, Yuriy and Anna, Ivan, Terje, Albert, Alberto, Narjes, Katie, Reidun, Martin, Patrick and so many others for creating a fantastic atmosphere at the department.  

Finally, I would not get this far without the constant support of my family. Thank you to my loving parents and sister for believing in me since the beginning. Thank you to my awesome grandparents who have regularly followed up with the status of the PhD, driving it forward. Most importantly, thank you to my wonderful wife, Helia, for your unconditional motivation and for helping me carry the burden of the PhD throughout its final and most crucial stages.  


%% LISTS (content, figures, tables)
\clearpage{\thispagestyle{empty}\cleardoublepage}
\tableofcontents
\clearpage{\thispagestyle{empty}\cleardoublepage}
\addcontentsline{toc}{chapter}{List of Figures}
\listoffigures
\clearpage{\thispagestyle{empty}\cleardoublepage}
\addcontentsline{toc}{chapter}{List of Tables}
\listoftables

% NOMENCLATURE
\clearpage{\thispagestyle{empty}\cleardoublepage}

\nomenclature[A]{EOR}{Enhanced Oil Recovery}
\nomenclature[A]{NTNU}{Norwegian University of Science and Technology}
\nomenclature[A]{SSW}{Synthetic sea water}
\nomenclature[A]{HyGreGel}{Hybrid Green nano-Gels}
\nomenclature[A]{HPAM}{Partially hydrolyzed polyacrylamide}
\nomenclature[A]{PAM}{polyacrylamide}
\nomenclature[A]{SP}{sub-project}
\nomenclature[A]{FN}{FunzioNano}
\nomenclature[A]{IPV}{Inaccessible pore volume}
\nomenclature[A]{PV}{pore volume}
\nomenclature[A]{RRF}{Residual resistance factor}
\nomenclature[A]{IGP}{Department of Geoscience and Petroleum at NTNU}
\nomenclature[A]{DMN}{Department of Materials and Nanotechnology at SINTEF Industry}
\nomenclature[A]{DP}{Department of Petroleum at SINTEF Industry}
\nomenclature[A]{DIP}{Department of Industrial Process Technology at SINTEF Industry}
\nomenclature[A]{wt.\%}{weight percent}
\nomenclature[A]{PEC}{Polyelectrolyte complexes}
\nomenclature[A]{DS}{dextran sulphate}
\nomenclature[A]{PEI}{polyethyleneimine}
\nomenclature[A]{POSS}{polyhedral oligomeric silsequioxanes}
\nomenclature[A]{PVS}{polyvinyl sulfonate}
\nomenclature[A]{ppm}{parts per million}
\nomenclature[A]{MDa}{Mega Daltons}
\nomenclature[A]{UV}{ultra violet}
\nomenclature[A]{inj}{injection}
\nomenclature[A]{pol}{polymer}
\nomenclature[A]{NP}{nanoparticles}


\nomenclature[S]{$\tau$}{degree of hydrolysis}
\nomenclature[S]{$y$}{molar concentration of carboxylate groups}
\nomenclature[S]{$x$}{molar concentration of the amide groups}
\nomenclature[S]{$\frac{c}{c_0}$}{relative fines concentration}
\nomenclature[S]{$\phi$}{porosity}
\nomenclature[S]{$x$}{distance}
\nomenclature[S]{$u$}{flowrate}
\nomenclature[S]{$T$}{temperature}
\nomenclature[S]{$C_x$}{concentration of x (or phase x)}
\nomenclature[S]{$S_i$}{saturation of phase $i$}
\nomenclature[S]{$u_i$}{volumetric flux of phase $i$}
\nomenclature[S]{$P$}{pressure}
\nomenclature[S]{$k$}{absolute permeability}
\nomenclature[S]{$k_{ri}$}{relative permeability to phase $i$}
\nomenclature[S]{$S_{or}$}{residual oil saturation}
\nomenclature[S]{$S_{wi}$}{irreducible water saturation}
\nomenclature[S]{$k^0_{ri}$}{endpoint relative permeability of phase $i$}
\nomenclature[S]{$\alpha_i$}{Corey exponent for phase $i$}
\nomenclature[S]{$C_{ai}$}{adsorbed concentrations per mass of the rock for phase $i$}
\nomenclature[S]{$\phi^{(i)}$}{porosity (acessible pore space) for nanopartilces or polymer}
\nomenclature[S]{$\rho_r$}{rock mass density}
\nomenclature[S]{$p(x,t;\tau)$}{age distribution of nanoparticles in solution at position and time}
\nomenclature[S]{$q(x,t;\tau)$}{age distribution of adsorbed nanoparticles at position and time}
\nomenclature[S]{$T_1$}{maximal age}
\nomenclature[S]{$\mu^p_w$}{water viscosity with only polymer present}
\nomenclature[S]{$\mu_w^{\max}$}{maximal possible water viscosity}
\nomenclature[S]{$a (C_p)$}{viscosity increase factor of polymer with no shear-thinning}
\nomenclature[S]{$s(C_p, u_w)$}{viscosity decrease factor due to shear thinning}
\nomenclature[S]{$h(C_n,C_p)$}{$h$-function}
\nomenclature[S]{$m(\tau)$}{weight function}
\nomenclature[S]{$\nu$}{interpolator, signifying degree of gelling}
\nomenclature[S]{$\lambda_i$}{mobility of phase $i$}
\nomenclature[S]{$f_i$}{fractional flow of phase $i$}
\nomenclature[S]{$A$}{cross sectional area}
\nomenclature[S]{$Q_w^{n+1}$}{volumetric injection rate for water in the time interval $\left[t^n,t^{n+1}\right]$}
\nomenclature[S]{$m^n$}{number of ages at time $t^n$}
\nomenclature[S]{$\omega$}{Todd-Longstaff mixing parameter}
\nomenclature[S]{$\mu_{p_\textit{eff}}$}{effective viscosity}
\nomenclature[S]{$\mu_m(C_p)$}{viscosity of a fully mixed polymer solution}

\addcontentsline{toc}{chapter}{Nomenclature}
\printnomenclature

