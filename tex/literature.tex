\chapter{Literature Review}
% -------------------------
%% QUOTE
\vspace*{\fill}
\epigraph{Those who don't know history are destined to repeat it.}%
{\textsc{Edmund Burke}}
\clearpage{\thispagestyle{empty}\cleardoublepage}
%%
%% Body of the chapter
%%%%%%%%%%%%%%%%%%%%%%


\section{Overview and properties of gel systems}

A gel-system, as referred to in this work, is a polymer solution that has gone through gelation to form a highly viscous and immobile material in the porous formation. The properties of gels in terms of their basic chemistry are discussed in this section, and general overview about polymers is introduced.

Polymer, as a word consists of two parts: “poly” (many) and “mer” (parts) . A polymer is a macro-molecule that is made of repetitive smaller molecules, referred to as monomers, which are either covalently or ionically bonded. Despite the simple concept of polymers, it was not accepted until late 1930’s, as Hermann Staudinger laboratory results were successful in synthesizing polymers, and obtained because of it the Noble Prize in Chemistry in 1953 \citep{Roberts1977}.

Polymers have different physical and mechanical properties compared to their original
monomers. This is due to the significant difference in the length to diameter ratio between the polymer and its original monomer constituents \citep{Ghosh2006}. Figure \ref{fig:polymer} shows a polymer molecule composed of small monomers.

\begin{figure}
    \centering
    \includegraphics[width=\textwidth]{img/fig/polymer.png}
    \caption{Polymer consisting of several attached monomers (circles) \citep{Ghosh2006}}
    \label{fig:polymer} % 2.1
\end{figure}

Figure \ref{fig:polymonomer} shows a synthetic polymer, referred to as polyacrylamide (PAM) which is one of the most applied polymers in conformance enhancement operations due to its low cost and its good viscosifying properties \citep{Kabir2001}. The monomer (left) can be repeated multiple times in the sequence of the polymer (right) to obtain the desired properties.

\begin{figure}
    \centering
    \includegraphics[width=\textwidth]{img/fig/polymonomer.png}
    \caption{Monomer: Amide (left) and polymer: polyacrylamide (right) \citep{Kabir2001}}
    \label{fig:polymonomer} % 2.2
\end{figure}

\subsection{Gel system types}

Gels can be divided into organic and inorganic. Inorganic gels include silicate-based, and Al(III)-based. Organic polymers can be classified in different ways, e.g. based on their origin, thermal response, mode of formation, line structure, physical properties, tacticity, and crystallinity \citep{Ghosh2006}. In the petroleum industry, organic polymers are often classified based on their origin. This includes natural polymers, i.e. biopolymers such as Xanthan and Scleroglucan \citep{Al-Muntasheri2012}, and synthetic polymers such as polyacrylamides, which are usually used in their partially hydrolyzed form, hereafter called HPAM \citep{Finch1992}. This report only addresses organic gel systems.

Polyacrylamide-based polymers have been used largely in water diversion operations throughout the history of the petroleum industry due to their low cost comparted with other gel-systems, and due to the good mechanical strength of the formed gel. 

\subsection{Polyacrylamide based gels}
Throughout the history of the petroleum industry, polyacrylamide-based gels were most commonly used in water diversion operations \citep{Al-Muntasheri2005, Ball1984} (Al-Muntasheri et al. 2005; Ball and Pitts 1984). In chemistry, acrylamide can be referred to as acrylic amide. Figure \ref{fig:acrylamide} shows the chemical formula of acrylamide (right) and the acryloyl group (left). The symbol “R” in the figure represents the possibility of having different group of atoms; in which the presence of \ce{NH2} makes the chemical structure defined as acrylamide.

\begin{figure}
    \centering
    \includegraphics[width=\textwidth]{img/fig/acrylamide.png}
    \caption{Acryloyl group (left) and acrylamide (right)}
    \label{fig:acrylamide} % 2.3
\end{figure}

PAM-solutions undergo hydrolysis, i.e., the reaction of the amide groups of the acrylamide solution with alkaline solutions resulting in carboxylate groups and ammonia. Hydrolysis of acrylamide is shown in Figure \ref{fig:amideHydrol}. The result is a partially hydrolyzed polyacrylamide, HPAM. The degree of hydrolysis, $\tau$ , can be defined as:

\begin{equation}
    \tau = \frac{y}{x+y}
\end{equation}							

where $y$ is the molar concentration of the yielded carboxylate groups and $x$ is the molar concentration amide groups \citep{Al-Muntasheri2012}.

\begin{figure}
    \centering
    \includegraphics[width=\textwidth]{img/fig/amideHydrol.png}
    \caption{Hydrolysis of amide groups at high pH conditions \citep{Al-muntasheri2008}}
    \label{fig:amideHydrol} % 2.4
\end{figure}

Partially hydrolysation of PAM is important as the negatively charged carboxylate groups are needed to initiate the gelation process in the presence of organic or inorganic cross-linkers. Al-Muntasheri, suggests that there must be at least 1 mol of carboxylate groups before crosslinking can take place. However, there exists an upper limit for hydrolysis that if it is exceeded, the formed gel might either get weaker or precipitate \citep{Al-Muntasheri2007}. When crosslinking is taking place, i.e. the gelation process has been initiated, there exists a threshold value at which the viscoelastic properties dominate the mechanical properties of the polymer solution. At this stage, which is referred to as the sol-gel transition period, the solution is better described as a gel \citep{Albonico1994}.

HPAM requires a cross-linker to undergo gelation. Normally, the gelation time can take several hours, i.e. the time for the gellant (polymer in aqueous state) to become a gel. \cite{Sydansk1993} Sydansk (1993) described a method in the literature to identify the gradual changes observed in the gellant until it becomes a gel, based on its physical behaviour.

These cross-linkers can be either organic or inorganic \citep{Al-Muntasheri2012}. The selection of either type is dependent on the specific type of application and reservoir conditions \citep{Ball1984}. The amount of the added crosslinker to the aqueous solution plays an important role as low amounts of it may result in long term stability issues, and high amounts arise the probability of having syneresis \citep{Sydansk1993, eggert1992}. Syneresis imposes a significant issue as the volume of the formed gel is reduced, secondary channel paths are created, resulting in a failure of the water-diversion application \citep{Al-Muntasheri2007}. However, in general, it has been observed that higher concentrations of cross-linkers result in more viscous gels as shown in Figure \ref{fig:crosslinkerConc}, given that the level of exceeding syneresis threshold is not reached.

\begin{figure}
    \centering
    \includegraphics[width=\textwidth]{img/fig/crosslinkerConc.png}
    \caption{Effect of increasing crosslinker (polyetheyleneimine, PEI) concentration on viscosity vs. time \citep{Al-Muntasheri2007}}
    \label{fig:crosslinkerConc} % 2.5
\end{figure}

Inorganic cross-linkers are often referred to as metal crosslinking agents. As the negatively charged (anions) carboxylate groups will chemically bond with the positively charged (cations) cross-linkers, the aforementioned gelation procedure will be initiated. This type of bonding, i.e., ionic bonding, is relatively a weaker type of bonding, as it precipitates at temperatures higher than 75~\celsius, compared with the covalent bond, formed in organic cross-linkers \citep{Al-Muntasheri2005}. Inorganic cross linkers include \ce{Al^3+}, \ce{Zr^4+}, \ce{Cr^3+}, where the latter is the most common cross-linker. Furthermore, \ce{Cr^3+} has the possibility to crosslink biopolymers such as Xanthan which is expensive and difficult to be implemented in large-scale water diversion projects \citep{Al-Muntasheri2012}. However, the application of \ce{Cr^3+} in several countries is limited due to environmental concerns. This will be discussed more in Section \what 2.4.

Organic cross linkers were introduced in the industry in order to overcome the limitation of temperature ranges\citep{Al-Muntasheri2005}. As previously mentioned, covalent bonds are much more stable compared to ionic bonds. Examples of organic cross linkers include aldehydes, phenol-formaldehyde and polyetheyleneimine. The polyacrylamide-phenol/formaldehyde system is a good example on an organically cross-linked gel, and has been reported to withstand the temperature of 121~\celsius~ for about 13 years. However, the use of this gel has been limited due to environmental concerns.

\section{Gel systems behaviour in porous media}

Several gel-systems have different chemical, and physical properties that are needed in the water-diversion applications. Basically, for in-depth water-diversion programs, it is appreciated to use a gel that blocks the high-permeable formation, flows easily through the porous rock before gelation takes place, forms an in-situ highly viscous material that dramatically minimizes the permeability, and shows minimal degradation signs when aged for several years. Retention, deposition, and adsorption are some of the critical parameters that affect the application of a particular gel-system. In this section, the aforementioned properties are discussed.

Particle Transport and Deposition in Porous Media. The behavior of the small-sized particles, e.g., the size of silicate particles to be injected in the reservoir, can be modeled using the classical fine particle transport \citep{Stavland2011}. They discussed several models regarding particle deposition such as Gruesbeck and Collins (1982), Bedrikovetsky et al. (2010), and Guedes et al. (2009). Figure \ref{fig:finesDeposition} shows the deposition of fine particles along the core length (x) using Gruesbeck and Collins simple model. The relative fines concentration, $c/c_0$, can be described as follows:

\begin{equation}
    \frac{c}{c_0} = e^{\frac{-\phi bx}{u}}
\end{equation}

where $\phi$ is the porosity, $b$ is a constant independent of the flowrate, $x$ is the distance, and $u$ is the flowrate. The values used in Figure \ref{fig:finesDeposition} are $\phi = 0.22$ and $b = 1*10^{-5}$ m/s \citep{Stavland2011}. The flowrate is varied as shown in Figure \ref{fig:finesDeposition}, and indicates that at low velocities, higher deposition of particles takes place.

\begin{figure}
    \centering
    \includegraphics[width=\textwidth]{img/fig/finesDeposition.png}
    \caption{Deposition of fines along the core length \citep{Stavland2011}}
    \label{fig:finesDeposition} % 2.6
\end{figure}

Retention can occur in two different ways, by rock adsorption of the injected polymer solution and possibly also by mechanical entrapment as shown in Figure \ref{fig:retention} \citep{Nabzar1996}. At high velocities, no bridging takes places which is likely due to the high viscous forces \citep{Stavland2011}. As the velocity is decreased, bridging is initiated. Eventually, accumulation takes place, and the pore-throat becomes plugged. In the literature, several authors reported laboratory experiments for gel-system solutions where effect of wettability, mobility reduction, and polymer injection flowrate on retention were examined\citep{Broseta1995, Cohen1986, Idahosa2016}.

\begin{figure}
    \centering
    \includegraphics[width=\textwidth]{img/fig/retention.png}
    \caption{Steps of retention at the grain/pore level \citep{Nabzar1996}}
    \label{fig:retention} % 2.7
\end{figure}



\section{Gel systems properties and effects}
There exist several factors that can alter the gelation time, the viscosity as well as the stability of the formed gel-system. These factors include temperature, salinity, pH, and the formation hardness. The main motivation for understanding these effects is to design and select a proper gel-system that is compatible with the reservoir conditions. This section is dedicated solely for polyacrylamide-based polymers properties and effects.

\subsection{Temperature effects} \label{sec:tempEffects}

Reservoir temperature is a major factor that influences the selection criteria of the gel-system to be used in the design of water conformance programs. As discussed earlier, for in-depth water diversion applications, gelation time should be of several weeks, its magnitude depending on characteristics of the reservoir and the nature of the job from an operational standpoint. One of the challenges encountered using Cr3+/HPAM-based polymers is the short gelation times above 60~\celsius~ \citep{Albonico1994}. The use of such gel-systems is possible for near wellbore treatments at low temperature conditions. However, if the temperature is slightly elevated, other considerations should be taken such as the use of un-hydrolyzed PAM and pre-cooling of the near wellbore region \citep{Albonico1994,Al-Muntasheri2012}. Table \ref{tab:gelTimevTemp} shows the gelation times vs temperature for \ce{Cr(acetate)3}/PAM gels.

\begin{table} 
\centering
\caption{Gelation times vs temperature for \ce{Cr(acetate)3}/HPAM gels \citep{Albonico1994}}
\label{tab:gelTimevTemp} % 2.1
\begin{tabular}{c c } 
\toprule
\textbf{Temperature} & \textbf{Gelation time}\\
~[\celsius] & [hour]\\
\midrule 
60   & 48\\
90   & 2\\ 
120   & $<0.1$\\ 

\bottomrule
\end{tabular}
\end{table}

Albonico et al., proposed the use of gelation-retarding additives in order to reach temperatures at the range of 150~\celsius. \ce{Cr^3+}/HPAM reacts to form a gel very fast as the temperature increases, which makes it not very useful in water-diversion applications. Several complexing agents have the ability to retard the gelation by capturing \ce{Cr^3+} ions in form of complexes that are not reactive with the polymer. However, after a certain period of time, \ce{Cr^3+} ions are released and the gelation is initiated. The rate at which \ce{Cr^3+} is released from the complexes is dependent on the complexing agent. Table \ref{tab:gelTimeHpam} shows gelation times of HPAM solutions with Cr3+ complexes at 90~\celsius. 

\begin{table} 
\centering
\caption{Gelation times of HPAM solutions with \ce{Cr^3+} complexes at 90~\celsius \citep{Albonico1994}}
\label{tab:gelTimeHpam} % 2.2
\begin{tabular}{l c } 
\toprule
\textbf{Cr(III) complex} & \textbf{Gelation time}\\
 & [hour]\\
\midrule 
\ce{Cr(NO3)3}  &                $< 0.02$    \\
\ce{Cr(acetate)3}  &            $< 0.12$    \\ 
\ce{K2Cr(glycolate)3}  &        1           \\ 
\ce{Cr(salicrylate)3}  &        1           \\
\ce{Cr(bipyridine)3(ClO4)3}  &  8           \\
\ce{Na3Cr(malonate)3}  &        48          \\

\bottomrule
\end{tabular}
\end{table}

In addition, it has been proven experimentally that extended delays can be obtained by adding non-complexing agents. A linear relationship between the gelling time and concentration has been found. 

Table \ref{tab:gelRetardAgent} presents the results obtained by Albonico et al. showing that the concentration of the non-complexing retarding ligands can significantly enhance the gelation time. However, it is important to note that adding excessive amounts of retarding ligands has an opposite effect as it can block and disable the initiation of gelation. This is because the \ce{Cr^3+}/ligand complexes become more stable compared to the Cr3+/polymer complexes. Therefore, extensive laboratory trials should be conducted before considering field-scale applications \citep{Albonico1994}.


\begin{table} 
\centering
\caption{Observed range of gelation times of \ce{Cr(malonate)3}/PAM-AMPS in SSW with different retarding agents \citep{Albonico1994}}
\label{tab:gelRetardAgent}  % 2.3
\begin{tabular}{l c c } 
\toprule

\multirow{2}{9em}{\textbf{Retarding Ligand}} & \multicolumn{2}{c}{\textbf{Gelation time (hours)}}\\
\cmidrule{2-3}
 & 90~\celsius & 120~\celsius\\
\midrule 
Gycolate      & 0.9 - 59  &     0.25 – 17      \\
Salicylate    & 0.9 - 27  &     0.25 – 17       \\ 
Malonate      & 0.9 - 215  &    0.25 - 42    \\ 

\bottomrule
\end{tabular}
\end{table}

\citet{Sydansk1993} reported as well that it is possible to obtain gels that can be applied to high temperature reservoirs reaching 126~\celsius using \ce{Cr^3+}-carboxylate complex with low molecular weight and HPAM with a low degree with hydrolysis. The carboxylate anion suggested by Sydansk is ``\textit{acetate}" its availability and low price as a commodity. Moreover, it is important to note that Sydansk’s experiments were based on near-wellbore treatments. 

According to Sydansk, for intermediate reservoir temperatures of about 60~\celsius, low HPAM concentration (1 wt. \%) restricted the formation of the gel. In addition, at low hydrolysis levels ($<0.1$), gelation was not observed. Concentrations of 2, 3, and 4 wt. \% were investigated in the experiments and showed formation of rigid gel according to the experiment specifications. The polymer had a molecular weight of 2 MDa, the degree of hydrolysis was 0.9 and the temperatures was 60~\celsius~ or less. The cross-liker was \ce{Cr(acetate)3}. Figure \ref{fig:viscSydansk} shows the development of the viscosity with time for different HPAM concentrations.

\begin{figure}
    \centering
    \includegraphics[width=0.75\textwidth]{img/fig/viscSydansk.png}    \caption{Viscosity vs. Time for 2 MDa HPAM (PA) in fresh water aged at 60~\celsius~\citep{Sydansk1993}}
    \label{fig:viscSydansk} % 2.8
\end{figure}

The general findings from the experiments were that as the HPAM concentration increases, the rate of gelation and the dynamic viscosity increase. Other findings include the unchanged rigidity and good stability of the tested gels for more than 730 days.

As has been discussed earlier \citep{Albonico1994}, the increased temperature directly affects the rate of gelation. In order to delay the gelation time, Sydansk concluded from several experiments where low molecular weight polymers (100,000 – 500,000 Da) should be used with ultra-low degree of hydrolysis, $< 0.1$. The phenomenon behind the applicability of using ultra-low hydrolysis degree is auto-hydrolysis that takes place only at elevated reservoir temperatures ($> 75$~\celsius) \citep{Sydansk1993, Fletcher2010}. The rate of auto-hydrolysis is dependent on both temperature and pH.

In order to enhance the thermal stability when HPAM based polymers are considered, other monomers can be co-polymerized with polyacrylamide solutions. Poly(vinylpyrrolidone-co-acrylamide) or PVP-AM is an example of a co-polymerized HPAM base. Polyvinylpyrrolidone improves the PAM base to undergo a lower degree of hydrolysis at elevated temperatures \citep{Stahl1988}. As a result, no precipitation takes place as the conversion of the amide groups into the carboxylate groups is minimized \citep{Al-Muntasheri2012}.

Figure \ref{fig:hyrolysisStahl} shows the degree of hydrolysis at 121~\celsius~ for different PVP-Am (60/40 w/w) and a commercial HPAM. As expected, HPAM-based polymer underwent full hydrolysis in less than 10 days of aging time, and was reported to be insoluble in synthetic seawater \citep{Stahl1988}. PVP-Am experienced a certain degree of hydrolysis and stabilized at 40 \% with no precipitation being reported. Additionally, PVP-AM at 150~\celsius~ did not show signs of precipitation despite reaching 60 \% degree of hydrolysis.

\begin{figure}
    \centering
    \includegraphics[width=0.75\textwidth]{img/fig/hyrolysisStahl.png}
    \caption{Degree of hydrolysis for different PVP-Am (60/40 w/w) and a commercial HPAM (PAm) \citep{Stahl1988}}
    \label{fig:hyrolysisStahl} % 2.9
\end{figure}

\subsection{Salinity and pH effects}
Gel systems are prepared using different sources of water. Water sources can be considered as either fresh or saline depending on where it has been obtained from; this can be nearby water wells, treated produced water, or seawater.

\citet{Al-Muntasheri2007} conducted several experiments where two different water sources were examined. Both were used in creating two different gel systems and the apparent viscosity of the polymer solution throughout the gelation period was recorded. The total dissolved solids in the brines were 1186 ppm, and 58348 ppm. Figure \ref{fig:almuntasheriSal} shows the effect of the difference in the salinity of brine. It is clear the effect of increasing the salinity will decrease the gelation time. However, gel systems that were prepared using more saline water exhibited a lower apparent viscosity compared to gel-systems prepared using fresh water.

\begin{figure}
    \centering
    \includegraphics[width=0.75\textwidth]{img/fig/almuntasheriSal.png}
    \caption{Effect of difference in salinity of brine – viscosity vs. time \citep{Al-Muntasheri2007}}
    \label{fig:almuntasheriSal} % 2.10
\end{figure}

In addition, Al-Muntasheri et al. investigated furthermore the possibility that if certain salt ions have a great effect on the gelation compared with others. Monovalent (\ce{K+} and \ce{Na+}) cations were tested as shown in Figure \ref{fig:almuntasheriKNa} which shows that potassium has a greater impact compared with sodium. This has been attributed to the greater charge density which is defined as the ionic charge per size of ion. Similarly, a divalent (\ce{Ca^2+}) cation and monovalent cation (\ce{K+}) were tested as shown in Figure \ref{fig:almuntasheriKCa} and resulted in potassium having a greater impact in extending the gelation time compared with calcium. Again, the same reason is applicable since the charge/size ratio of potassium is only half that of calcium. Therefore, it can be concluded that the gelation time is dependent on the charge/size ratio, rather than the valence of the atom.

\begin{figure}
    \centering
    \includegraphics[width=0.75\textwidth]{img/fig/almuntasheriKNa.png}
    \caption{Effect of K and Na cations on gelation time – gelation time vs. ion concentration (PAtBA=A copolymer of acrylamide and tert-butyl acrylate) \citep{Al-Muntasheri2007}}
    \label{fig:almuntasheriKNa} % 2.11
\end{figure}

\begin{figure}
    \centering
    \includegraphics[width=0.75\textwidth]{img/fig/almuntasheriKCa.png}
    \caption{Effect of K and Ca cations on gelation time – gelation time vs. ion concentration \citep{Al-Muntasheri2007}}
    \label{fig:almuntasheriKCa} % 2.12
\end{figure}

The effect of initial pH has been investigated by Al-Muntasheri et al. and the evolution of the viscosity with time was recorded. Adjusting the pH level of the solution could be done by adding variable amount of either HCL (acid) or NaOH (base). Results in Figure \ref{fig:almuntasheriAcidicpH} showed that at very acidic conditions (pH = 3.1) the gel viscosity increased and suddenly collapsed at around 1.3 hours. Furthermore, a less acidic condition (pH = 6.2), a period of stabilization was observed but gel breakage took place at 1.9 hours. 

Following experiments resulted in a conclusion that a basic pH of at least a value of pH = 8 should be considered when designing gel-systems based on PHPA. Figure \ref{fig:almuntasheriBasicpH} shows that the formed gel is stable (opposed to what observed at acidic pH), and the gelation time is dependent on the temperature as discussed in \ref{sec:tempEffects} \citep{Al-Muntasheri2007}.

\begin{figure}
    \centering
    \includegraphics[width=0.75\textwidth]{img/fig/almuntasheriAcidicpH.png}
    \caption{Effect of acidic pH (3.1 and 6.2) at constant temperature – viscosity vs. time \citep{Al-Muntasheri2007}}
    \label{fig:almuntasheriAcidicpH} % 2.13
\end{figure}

\begin{figure}
    \centering
    \includegraphics[width=0.75\textwidth]{img/fig/almuntasheriBasicpH.png}
    \caption{Effect of basic pH (8.3) at different temperatures – viscosity vs. time \citep{Al-Muntasheri2007}}
    \label{fig:almuntasheriBasicpH} % 2.14
\end{figure}

\subsection{Hardness effects (concentration of divalent ions)}

HPAM-based polymers have the tendency to precipitate, i.e., they can undergo de-gelation in formations with elevated temperatures with high hardness levels. This is due to the presence of divalent ions, e.g., \ce{Mg^2+}, \ce{Sr^2+}, \ce{Ba^2+}, and \ce{Ca^2+}, where the latter at equal molar levels, has a greater impact on gel stability. Reservoirs with temperatures less than 75~\celsius, the HPAM-based polymers were observed to be stable regardless of the hardness level \ref{Moradi-Araghi1987}. However, at temperatures higher than 75~\celsius, the amide groups in the polymer experienced extensive hydrolysis, due to the reaction with polyvalent ions present in the formation, which results in weakening the 3D gel network \citep{Al-Muntasheri2012, Stahl1989}. This goes in line with results from experiments reported by \citet{Davison1982} where 140 different HPAM-based polymers were examined at 90~\celsius. Most of them showed precipitation during the several weeks test period.


The situation is more severe when the formation contains higher concentration of hard brines. As the interaction between the amide groups (negative charge) and the divalent cations (positive charge) takes place, precipitation happens in the reservoir causing several negative impacts. Namely, the reduction in mechanical gel strength which leads to the failure of the 3D gel-network, and blockage of the flow in the porous media, leading to a complete shutdown in the injection operations. Carbonate-based gelation retarders can be found in several commercial polymer packages; when they are mixed with water containing high hardness (divalent ions concentration) levels, insoluble carbonate-based salts precipitate in the formation. In case of precipitation, plugging of formation might take place which results in a failure of the water diversion-program, as well as causing injectivity problems leading to a complete shut-down of the operation due to significant pressure increase \citep{Al-Muntasheri2012}.

The results of \citet{Moradi-Araghi1987} can be taken as a simple reference for brine hardness limits for polyacrylamide-based polymers. Table \ref{tab:formHardnessLim} shows the concentrations of divalent ions in brine after which the gel is no longer stable. For formation brines containing less than 20 ppm, the gels are stable at elevated temperatures of 204~\celsius. This shows the importance of formation hardness of the formation.


\begin{table} 
\centering
\caption{Temperature vs. Formation hardness limit for HPAM \citep{Moradi-Araghi1987}}
\label{tab:formHardnessLim} % 2.4
\begin{tabular}{c c } 
\toprule
\textbf{Temperature} & \textbf{Formation hardness limit}\\
~[\celsius] & [ppm]\\
\midrule 
75   & 2000\\
88   & 500\\ 
96   & 270\\ 
204   & 20\\ 
\bottomrule
\end{tabular}
\end{table}

Actual precipitation of the gel-system (due to hardness levels) is governed by the cloud point temperature. As the temperature of an aqueous solution containing divalent ions is slowly increased, at a certain temperature, the solution will become cloudy; this is referred to as the cloud point. Increasing the temperature to a further extent will lead into precipitation of the gel system. The cloud point can be considered as the upper limit of the temperature at which the gel-system will undergo precipitation, and therefore, all polyacrylamide based polymers design should be based on lower values \citep{Moradi-Araghi1987, Stahl1988}.

\citet{Moradi-Araghi1987} conducted several experiments to understand the dependency of the cloud point on \textit{brine hardness level, polymer concentration} and \textit{molecular weight}.

\textbf{Hardness level.} The cloud point is strongly affected by the hardness level of the brine and the degree of hydrolysis. The temperature at which the cloud point is visible, decreases as both hardness level, and degree of hydrolysis increase. Going from a hardness level of 1 to 10000 ppm (\ce{Ca^2+} + \ce{Mg^2+}), the cloud point dropped from 200~\celsius~ to 170~\celsius~ with 0\% initial degree of hydrolysis, and from 200~\celsius~ to 20~\celsius~ with 93\% degree of hydrolysis.

\textbf{Polymer concentration.} As the polymer concentration increases, the cloud point decreases. It has been shown experimentally that the cloud-point is less affected by the polymer concentration compared with divalent ion concentration. In addition, the cloud time is assumed to be independent at concentrations below 0.25\%, in which most of the applications of polyacrylamides are used at or even below that level in field applications.

\textbf{Molecular weight} seems to influence the cloud point temperature much less compared to the aforementioned investigated parameters. In addition, inconsistent behavior has been obtained from the results when varying the molecular weight. It has been assumed that the reported molecular weights from the vendors might be incorrect.

Determination of the precipitation time, is conducted by plotting the could point temperature versus the hydrolysis level of the selected polymer at different brine hardness levels. As the formation hardness and temperature are usually known, hydrolysis level is the unknown. Next, a plot of hydrolysis level versus time is used to determine the precipitation time of the selected partially hydrolyzed polymer \citep{Moradi-Araghi1987}.

As discussed earlier, Davison and Mentzer experiments showed that most of the polyacrylamide based polymers precipitated after several weeks. However, polyvinylpyrrolidone is a polyacrylamide based polymer that withstood the temperature at 90~\celsius. Polyvinylpyrrolidone lacks the required viscosity strength for conformance control to be used in IOR operations, however \citep{Stahl1988}.

\section{Environmental considerations}
Injection of chemicals, i.e. polymers in case of in-depth water diversion applications, imposes certain risks related to environmental concerns. In the petroleum industry, chemicals are usually classified by colour codes based on their eco-toxicity. In Norway, four colour codes were suggested by "Klima- og forurensingsdirektoratet" (Klif), an Norwegian directorate that is responsible for regulating the chemical discharge permits. Figure \ref{fig:envColors} shows the definitions of each colour code \citep{Norskolje2018}.

\begin{figure}
    \centering
    \includegraphics[width=\textwidth]{img/fig/envColors.png}
    \caption{Toxicity color definitions based on Klif classification \citep{Norskolje2018}}
    \label{fig:envColors} % 2.15
\end{figure}



































