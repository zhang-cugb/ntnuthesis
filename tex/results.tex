\chapter{Results}\label{chap:results}
% -------------------------
%% QUOTE
\vspace*{\fill}
\epigraph{If you're doing an experiment, you should report everything that you think might make it invalid — not only what you think is right about it; other causes that could possibly explain your results; and things you thought of that you've eliminated by some other experiment, and how they worked — to make sure the other fellow can tell they have been eliminated.}%
{\textit{Surely You're Joking, Mr. Feynman!, p. 341}\\ \textsc{Richard Feynman}}
\clearpage{\thispagestyle{empty}\cleardoublepage}
%%
%% Body of the chapter
%%%%%%%%%%%%%%%%%%%%%%
\section{Nanogels from polyelectrolyte complexes}

% 4.1.1	Gels with Cr3+ as cross linker
% Three different polymers (all HPAM) were used in the experiments as summarised in Table 4.1. The product names, the producers and the approximate molecular weights are given. The concentrations used and the lowest concentration for gel formation are also given. The polymers were always dissolved in synthetic sea water, SSW (cf. Table 5.2). The exact molecular weight of Flopaam 5115 VHM is not known but assumed to be in the order of 12 - 15 MDa.

% Table 4.1	Polymers used in the experiments.
% Name	Producer	MW	Conc. used (wt. %)	Gels at
% Alcoflood 254 S	BASF	0.5 MDa	0.5, 1.0 and 2.0	2 wt. %
% Alcomer 24 UK	BASF	6 MDa	0.25, 0.5, 1.0 and 2.0	≥ 0.5 wt. %
% Flopaam 5115 VHM	SNF Floerger	12+ MDa	0.5, 1.0 and 2.0	≥ 0.5 wt. %
% As seen in Table 4.1 gel was only formed with Alcoflood 254 S for the highest concentration tested. However, it is possible that gels could have been formed at lower concentrations in the interval between 1 wt. % and 2 wt. %. Alcomer 24 UK formed gel at 0.5 wt. % but not at 0.25 wt. %. It is possible that the highest molecular weight polymer Flopaam 5115 VHM could have formed gels at concentrations lower than 0.5 wt. %.

% In order to get data for testing of the simulator developed for transport of nanoparticles and polymer, and with a functionality of time delayed gel formation (Chapter 6), some gel formation experiments were the polymer concentration was varied from 0.25 wt. % to 1 wt. % and the concentration of Cr3+ was varied from 22 ppm to 113 ppm. Alcomer 24 UK was used as polymer. For concentrations higher than 22 ppm the polymer viscosity was apparently not dependent on the cross-binder concentration (within the accuracy of the measured viscosities as discussed above). The viscosities of the solutions were measured directly after preparation and after one day of aging. Figure 4.2 shows viscosity as function of shear rate for three polymer concentrations. 

 
% Figure 4.2	Viscosity as function of shear rate and polymer concentration for Alcomer 24 UK.
% Figure 4.3 shows an example of a reaction model designed based on viscosities measured at a shear rate of 12.7 s-1 for the fresh made solutions and after one day of aging. The blue curve corresponds to the unreacted polymer, i.e. data taken from Figure 4.2. It also corresponds to the viscosities for aging times less than 50 days with reference to Figure 4.1 (although some viscosity increase was seen in this period). The red curve corresponds to maximum viscosities for the solutions, corresponding to the value after 60 days in Figure 4.1 that is only valid for a single polymer concentration. The green curve is for aging time laying between no gelling and complete gelling. Again, with reference to Figure 4.1 the green curve would be valid for an aging time of 55 days. In the reaction model, it is assumed that the viscosity increases linearly with time in the gelling period.

% Again, as mentioned above, the reaction model was solely made to have some reality based data to be used for testing of the simulator model. 
 
% Figure 4.3	Reaction model for Alcomer 24 UK.
