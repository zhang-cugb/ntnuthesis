\chapter{Numerical simulation of time delayed gelling } \label{chap:simulation}
% -------------------------
%% QUOTE
\vspace*{\fill}
\epigraph{... in real life mistakes are likely to be irrevocable. Computer simulation, however, makes it economically practical to make mistakes on purpose. If you are astute, therefore, you can learn much more than they cost. Furthermore, if you are at all discreet, no one but you need ever know you made a mistake.}%
{\textit{Natural Automata and Useful Simulations}\\ \textsc{John H. Mcleod}}
\clearpage{\thispagestyle{empty}\cleardoublepage}
%%
%% Body of the chapter
%%%%%%%%%%%%%%%%%%%%%%

The following is a description of a model, developed at SINTEF Energy, for incompressible two-phase Darcy flow (water and oil) where the water phase can contain two solvents, nanoparticles and polymers. The novel feature of the presented model is that the nanoparticles can act as time-delayed cross-linkers, postponing the formation of gel. The given model has also been discretized and implemented as a c-code. A documentation of input and output of this program is also given.

\subsection{Background}
The presented model is spatially 1-dimensional with rate-controlled injection. Thus, the model is most appropriate for simulation of core flooding experiments with specified injection rates. At present, the given version of the model does not contain capillary forces nor gravity. However, the formulation of transport properties can be generalized to more spatial dimensions, to include capillary forces and gravity, and to include the possibility of pressure-controlled boundary conditions. 

The model also contains other features (in addition to age tracking of nanoparticles) such as adsorption and possible desorption of nanoparticles and polymers, inaccessible pore space for nanoparticles and polymers, shear thinning, and absolute permeability reduction as a function of adsorbed polymer concentration. 

The water viscosity at a given location and time is a function of polymer concentration, nanoparticle concentration, the local age profile of the nanoparticles, as well as the rate. With no nanoparticles present, the formulation of the water viscosity as a function of polymer concentration corresponds to the fully mixed Todd-Longstaff formulation. When gelling is possible (i.e. when sufficiently aged nanoparticles and polymers are present), the water viscosity is interpolated between its value corresponding to no gelling and its maximal possible value (maximal gelling).  For generality, several of the input parameters defining water viscosity in the developed code are table based, allowing for flexibility in the definition of rheological properties of the water.

The numerical formulation of the presented model is standard upstream implicit Euler for the transport of water, oil, polymers, and nanoparticles, while the recalculation of the nanoparticle age distributions is done after the implicit transport equations have converged. This recalculation applies a relatively novel method \citep{Flatten2008} by treating the upstream terms explicit and the downstream terms implicit. This approach givens stability and limited dispersion. 

Since the model is spatially 1-dimensional, the Jacobian matrix in the Newton iteration has a block structure enabling a non-iterative robust and effective linear solver. Indeed, numerical simulations demonstrate robust stability due to the implicit formulation for fluid transport, and the simulator is numerically effective allowing for short timesteps in order to limit numerical dispersion inherent in the implicit formulation.

\subsection{Defining equations}
We consider an immiscible two-phase (water and oil) 1D model with injection of nanoparticles and polymer in solution with water, where capillary and gravitational forces are ignored. We will assume that polymer and sufficiently old nanoparticles forms gel, increasing the water viscosity significantly. The boundary conditions are given by specifying an (time dependent) input rate. 

\subsubsection{Conservation laws}
Mass conservation for oil and water read

\begin{equation}
    \phi\frac{\partial S_i}{\partial t}+\frac{\partial u_i}{\partial x} = 0,\quad i = o, w
\end{equation}

where $\phi$ is the constant porosity (porosity for oil and water), $S_i=S_i(x,t), i = o,w$ are the phase saturations and $u_i, i=o,w$ are the phase volumetric fluxes. 